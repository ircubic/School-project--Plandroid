\documentclass[]{report}
\usepackage[usenames,dvipsnames]{color}
\usepackage{listings}
\usepackage{hyperref}
\usepackage{hypcap}


% Set up some colors
\definecolor{gray92}{gray}{.92}
\definecolor{gray75}{gray}{.75}
\definecolor{gray45}{gray}{.45}

% Set up some PDF options and reference coloration
\hypersetup{
    pdftitle={Mobile Applications - Designing an Event App}
    pdfauthor={Daniel E. Bruce},     % author
    colorlinks=true,       % false: boxed links; true: colored links
    linkcolor=red,          % color of internal links
    citecolor=black,        % color of links to bibliography
    filecolor=magenta,      % color of file links
    urlcolor=cyan           % color of external links
}

% listings settings
\lstset{
  breaklines=true,
  framerule=0.5pt,
  linewidth=\textwidth,
  numbers=left,
  showstringspaces=false
}

\lstdefinestyle{console}
{
  numbers=none,
  basicstyle=\bf\ttfamily,
  backgroundcolor=\color{gray92},
  frame=lrtb,
}

\lstdefinestyle{java}
{
  keywordstyle=\color{Maroon}\bfseries\emph,
  frame=lines,
  basicstyle=\small\ttfamily,
  commentstyle=\color{ForestGreen},
  stringstyle=\color{Mulberry}
}

\lstset{language=Java,style=java}

% Use "normal" paragraph separation
\setlength{\parskip}{1.3ex plus 0.2ex minus 0.2ex}
\setlength{\parindent}{0pt}

% Nicer margin comments
\let\oldmarginpar\marginpar
\renewcommand\marginpar[1]{\-\oldmarginpar[\raggedleft\footnotesize #1]%
{\raggedright\footnotesize #1}}

%
% Document begins here
%
\begin{document}
\title{Mobile Applications - Designing an Event App}
\author{Daniel E. Bruce}
\date{\today}
\maketitle

\begin{abstract}
  \begin{description}
    \item[Background] \hfill \\
      --
    \item[Results] \hfill \\
      --
  \end{description}
\end{abstract}

\tableofcontents

\chapter{Introduction}

\section{Background}

There are already solutions for planning events socially, through Facebook
Events and similar platforms, there are also solutions for location-based
services. However, there are no services that integrate the two in a seamless
manner, and both types of services have traditionally had separate focuses.

In addition, the planner still has to conduct several conversations with people
if the events that are planned rely heavily on the availability of the invited
participants, as there are currently no good solutions for automating schedule
conflict resolution.

\section{Motivation}

The aim of this report is to design a social event planning application
that focus on low-key, day-to-day events, utilizes already existing social
networks to give recommendations, uses the information stored in the networks to
streamline event creation, and a map-based view. The type of events planned are
intended to be the personal, small-scale variants, like ``we're going to lunch,
where do we want to go?'', ``I'm having a movie get together, who wants to
come'', as opposed to big-scale events like concerts, movie showings and
similar.

The motivation for this is to explore what benefits, if any, people can get from
using mobile devices to streamline inherently social day-to-day tasks, and if
modern tools like the social graph and recommendation systems can improve on how
they are done, and make friends feel more intimately connected. The report will
especially focus on the schedule negotiation aspect of planning events.

\section{Context}
This report was written in conjunction with the Mobile Applications course at
H\o{}gskolen i \O{}stfold\cite{site:mobapp}.

\section{Previous and related work}

\subsubsection{Similar solutions}

\begin{description}
\item[Belugapods] A group-based messaging service that allows conversation
  within ``pods'', groups you can subscribe to, allowing for simple group
  conversations or planning.\cite{site:belugapods}
\item[Facebook (events)] Facebook has support for planning events, although as a
  peripheral functionality. It does not support showing events on a map
  directly, though you can list an entry from Places as destination, which you
  can click through to for a map.\cite{site:facebook}
\item[Facebook (places)] Places allows location-based check-ins, and shows the
  place on a map, but does not support tighter integration with events.\cite{site:facebook}
\item[Foursquare] A game that uses location-based check-ins, which gives you
  badges for accomplishing certain tasks or going to certain places. It also
  allows you to give tips about places.\cite{site:foursquare}
\item[Gowalla] Similar to foursquare.\cite{site:gowalla}\marginpar{Fill out more here}
\end{description}

\chapter{Methods}

\section{Phases}

The method I intend to use in designing this application will go through the
phases of Discovery, Design and Evaluation. The phases will be described in
detail in further sections, but a quick summary is given:

\begin{description}
\item[Discovery] In the discovery phase, data is gathered to attempt to get an
  overview of what's already out there, what problems existing solutions have,
  what the potential users of the application want, and other useful info.
\item[Design] In the design phase, the data gathered so far is processed, and a
  prototype for the application (of any level of completeness) is created from
  it.
\item[Evaluation] The design is evaluated through user tests and/or other,
  heuristic, tests. If these tests indicate that the current prototype is not
  good enough, we return to the design phase to make another.
\end{description}

\subsection{Discovery}

The discovery phase is when the initial data regarding the application to be
designed is gathered. In this phase, I will find similar solutions to the one I
am designing, looking for their strengths and weaknesses. I will also perform
interviews focused on answering questions about the application that are
unclear, or just to get ideas on how to solve certain problems.

Once I have gathered enough info, I will process the information I have, and
create an initial requirements document documenting my initial ideas for the
application, then move onto the first design phase.

\subsection{Design}

In the design phase, I will create prototypes for the application, based on the
latest requirements document. The prototypes created will all be paper
prototypes, as there is no time to make a functional prototype, and many of the
technologies inherent in my application take a lot of time and resources to
develop to even a prototype level.

\subsubsection{Personas}

As a help for the design process, I have created three personas that describes
main sections of the user base I'm expecting for my application. These will be a
help when evaluating requirements and design decisions.

\begin{enumerate}
\item Richard is a 21-year old college student, studying Economy. He has decent
  computer skills, and uses Facebook like most of his friends, but doesn't spend
  a lot of time on it. He recently got a smart-phone running Android, and uses a
  small selection of apps like Facebook, Email, and some games, although he
  still uses his phone most for texting and calling. In his spare time he hangs
  out with fellow students, goes to the gym to stay in shape, and likes to bike.
\item Ashley is an 18-year old college student, studying Computer Science. She
  is considered a power-user, even for a CS-student, and is a so-called Early
  Adopter, that likes to try out the newest shiny social networks and
  services. She has had an Android phone since they came out, and fills her
  phone memory with as many apps as she can find, swapping them out at a rapid
  pace. In her spare time she likes to hang out with friends, arrange gaming
  nights, and likes to go rock-climbing occasionally.
\item Michael is a 34-year old working in the entertainment industry as a
  designer. He uses a MacBook for his day-to-day work, and values good design in
  his applications. Despite this, he owns an Android phone, because it was
  cheaper. As a father of two, he spends a lot of his spare time with his
  family, but also likes getting together with his buddies from time to time to
  have lunch or a beer.
\end{enumerate}

\subsubsection{Evaluation}

In the evaluation phase, I will use the prototypes created in the Design phase,
then run them through user tests. These user tests will focus on running the
users through various tasks, observing their behaviour, how intuitive the
design is, as well as if displayed screens match user's expectations. These
tests will be designed to answer a set of questions determined before each test
is created.

In addition, informal interview or question sessions may be done after these
user tests to answer any questions that might have arisen during tests, as well
as any that have been left unanswered from the initial set of questions.

If the evaluation phase points out that there are problems with the current
design, and there is time left, I will return to the design phase, otherwise the
design is evaluated against the final evaluation criteria.

\section{Final evaluation criteria}

The final evaluation criteria will be based on the user tests on the final
prototype, as well as aggregated data from all the user tests that have been
run. They will look at how well the user likes the social features, and if they
perceive that they would make the job of planning events easier for them. In
addition, the design will be evaluated for usability concerns, and privacy
concerns.

\section{Frameworks}

\section{Process - motivated progress}

What?

\chapter{Results}

\section{Status of application / prototype}

\section{Final evaluation results}

\chapter{Discussion}

\section{Success(es)}

\section{Flaw(s)}

\section{What did I learn}

\section{What would I do different}

\section{Next steps - future work}

\section{References}

\bibliography{report}{}
\bibliographystyle{plain}

%Scientific references
%Previous work
%Web sites / projects / products

\section{Conclusion}

\chapter{Appendices}

Will be provided as separate tex documents and put afterwards, these headers are
just for reminders

\section{Code selections}

\section{Transcripts}

\section{Requirement documents}

\section{Paper prototype(s)}

Mockups\cite{site:mockups}

\end{document}
