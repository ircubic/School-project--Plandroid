\documentclass[]{report}
\usepackage[usenames,dvipsnames]{color}
\usepackage{listings}
\usepackage{hyperref}
\usepackage{hypcap}


% Set up some colors
\definecolor{gray92}{gray}{.92}
\definecolor{gray75}{gray}{.75}
\definecolor{gray45}{gray}{.45}

% Set up some PDF options and reference coloration
\hypersetup{
    pdftitle={Mobile Applications - Designing an Event App}
    pdfauthor={Daniel E. Bruce},     % author
    colorlinks=true,       % false: boxed links; true: colored links
    linkcolor=red,          % color of internal links
    citecolor=black,        % color of links to bibliography
    filecolor=magenta,      % color of file links
    urlcolor=cyan           % color of external links
}

% listings settings
\lstset{
  breaklines=true,
  framerule=0.5pt,
  linewidth=\textwidth,
  numbers=left,
  showstringspaces=false
}

\lstdefinestyle{console}
{
  numbers=none,
  basicstyle=\bf\ttfamily,
  backgroundcolor=\color{gray92},
  frame=lrtb,
}

\lstdefinestyle{java}
{
  keywordstyle=\color{Maroon}\bfseries\emph,
  frame=lines,
  basicstyle=\small\ttfamily,
  commentstyle=\color{ForestGreen},
  stringstyle=\color{Mulberry}
}

\lstset{language=Java,style=java}

% Use "normal" paragraph separation
\setlength{\parskip}{1.3ex plus 0.2ex minus 0.2ex}
\setlength{\parindent}{0pt}

%
% Document begins here
%
\begin{document}
\title{Mobile Applications - Designing an Event App}
\author{Daniel E. Bruce}
\date{\today}
\maketitle

\begin{abstract}
  \begin{description}
    \item[Background] \hfill \\
      --
    \item[Results] \hfill \\
      --
  \end{description}
\end{abstract}

\tableofcontents

\chapter{Introduction}

\section{Background}

There are already solutions for planning events socially, through Facebook
Events and similar platforms, there are also solutions for location-based
services. However, there are no services that integrate the two in a seamless
manner, and both types of services have traditionally had separate focuses.

\section{Motivation}

The motivation of this report is to design a social event planning application
that focus on low-key, day-to-day events, utilizes the social graph to give
recommendations, and a map-based view. The type of events planned are intended
to be the personal, small-scale variants, like ``we're going to lunch, where do
we want to go?'', ``I'm having a movie get together, who wants to come'', as
opposed to big-scale events like concerts, movie showings and similar.

The reason for this is to explore what benefits, if any, people can get from
using mobile devices to streamline inherently social day-to-day tasks, and if
modern tools like the social graph and recommendation systems can improve on how
they are done, and make friends feel more intimately connected.

\section{Context}

This report was written in conjunction with the Mobile Applications course at
H\o{}gskolen i \O{}stfold\cite{site:mobapp}.

\section{Previous and related work}

\subsection{Similar solutions}


\begin{description}
\item[Belugapods] A group-based messaging service that allows conversation
  within ``pods'', groups you can subscribe to, allowing for simple group
  conversations or planning.\cite{site:belugapods}
\item[Facebook (events)] Facebook has support for planning events, although as a
  peripheral functionality. It does not support showing events on a map
  directly, though you can list an entry from Places as destination, which you
  can click through to for a map.\cite{site:facebook}
\item[Facebook (places)] Places allows location-based check-ins, and shows the
  place on a map, but does not support tighter integration with events.\cite{site:facebook}
\item[Foursquare] A game that uses location-based check-ins, which gives you
  badges for accomplishing certain tasks or going to certain places. It also
  allows you to give tips about places.\cite{site:foursquare}
\item[Gowalla] Similar to foursquare TODO.\cite{site:gowalla}
\end{description}

\chapter{Methods}

\section{Discovery/Design/Evaluation}

\subsection{Personas}

As a help for the design process, I have created three personas that describes
main sections of the user base I'm expecting for my application. These will be a
help when evaluating requirements and design decisions.

\begin{enumerate}
\item Richard is a 21-year old college student, studying Economy. He has average
  computer skills, and uses Facebook like most of his friends, but doesn't spend
  a lot of time on it. He recently got a smart-phone running Android, and uses a
  small selection of apps like Facebook, Email, and some games, although he
  still uses his phone most for texting and calling. In his spare time he hangs
  out with fellow students, goes to the gym to stay in shape, and likes to bike.
\item Ashley is an 18-year old college student, studying Computer Science. She
  is considered a power-user, even for a CS-student, and is a so-called Early
  Adopter, that likes to try out the newest shiny social networks and
  services. She has had an Android phone since they came out, and fills her
  phone memory with as many apps as she can find, swapping them out at a rapid
  pace. In her spare time she likes to hang out with friends, arrange gaming
  nights, and likes to go rock-climbing occasionally.
\item Michael is a 34-year old working in the entertainment industry as a
  designer. He uses a MacBook for his day-to-day work, and values good design in
  his applications. Despite this, he owns an Android phone, because it was
  cheaper. As a father of two, he spends a lot of his spare time with his
  family, but also likes getting together with his buddies from time to time to
  have lunch or a beer.
\end{enumerate}

\section{Final evaluation criteria}

\section{Frameworks}

\section{Process - motivated progress}

\chapter{Results}

\section{Status of application / prototype}

\section{Final evaluation results}

\chapter{Discussion}

\section{Success(es)}

\section{Flaw(s)}

\section{What did I learn}

\section{What would I do different}

\section{Next steps - future work}

\section{References}

\bibliography{report}{}
\bibliographystyle{plain}

%Scientific references
%Previous work
%Web sites / projects / products

\section{Conclusion}

\chapter{Appendices}
% Will be provided as separate tex documents and put afterwards, this is just
% for reminders
\section{Code selections}

\section{Transcripts}

\section{Requirement documents}

\section{Paper prototype(s)}

\end{document}
