\documentclass[]{report}
\usepackage[usenames,dvipsnames]{color}
\usepackage{listings}
\usepackage{hyperref}
\usepackage{hypcap}


% Set up some colors
\definecolor{gray92}{gray}{.92}
\definecolor{gray75}{gray}{.75}
\definecolor{gray45}{gray}{.45}

% Set up some PDF options and reference coloration
\hypersetup{
    pdftitle={Mobile Applications - Designing an Event App}
    pdfauthor={Daniel E. Bruce},     % author
    colorlinks=true,       % false: boxed links; true: colored links
    linkcolor=red,          % color of internal links
    citecolor=black,        % color of links to bibliography
    filecolor=magenta,      % color of file links
    urlcolor=cyan           % color of external links
}

% listings settings
\lstset{
  breaklines=true,
  framerule=0.5pt,
  linewidth=\textwidth,
  numbers=left,
  showstringspaces=false
}

\lstdefinestyle{console}
{
  numbers=none,
  basicstyle=\bf\ttfamily,
  backgroundcolor=\color{gray92},
  frame=lrtb,
}

\lstdefinestyle{java}
{
  keywordstyle=\color{Maroon}\bfseries\emph,
  frame=lines,
  basicstyle=\small\ttfamily,
  commentstyle=\color{ForestGreen},
  stringstyle=\color{Mulberry}
}

\lstset{language=Java,style=java}

% Use "normal" paragraph separation
\setlength{\parskip}{1.3ex plus 0.2ex minus 0.2ex}
\setlength{\parindent}{0pt}

%
% Document begins here
%
\begin{document}
\title{Mobile Applications - Designing an Event App}
\author{Daniel E. Bruce}
\date{\today}
\maketitle

\begin{abstract}
  \begin{description}
    \item[Background] \hfill \\
      --
    \item[Results] \hfill \\
      --
  \end{description}
\end{abstract}

\tableofcontents

\chapter{Introduction}

\section{Background}

There are already solutions for planning events socially, through Facebook
Events and similar platforms, there are also solutions for location-based
services. However, there are no services that integrate the two in a seamless
manner, and both types of services have traditionally had separate focuses.

\section{Motivation}

The motivation of this report is to design a social event planning application
that focus on low-key, day-to-day events, utilizes the social graph to give
recommendations, and a map-based view. The type of events planned are intended
to be the personal, small-scale variants, like ``we're going to lunch, where do
we want to go?'', ``I'm having a movie get together, who wants to come'', as
opposed to big-scale events like concerts, movie showings and similar.

The reason for this is to explore what benefits, if any, people can get from
using mobile devices to streamline inherently social day-to-day tasks, and if
modern tools like the social graph and recommendation systems can improve on how
they are done, and make friends feel more intimately connected.

\section{Context}

This report was written in conjunction with the Mobile Applications course at
H\o{}gskolen i \O{}stfold\cite{site:mobapp}.

\section{Previous and related work}

http://belugapods.com/

\chapter{Methods}

\section{Discovery/Design/Evaluation}

\section{Final evaluation criteria}

\section{Frameworks}

\section{Process - motivated progress}

\chapter{Results}

\section{Status of application / prototype}

\section{Final evaluation results}

\chapter{Discussion}

\section{Success(es)}

\section{Flaw(s)}

\section{What did I learn}

\section{What would I do different}

\section{Next steps - future work}

\section{References}

\bibliography{report}{}
\bibliographystyle{plain}

%Scientific references
%Previous work
%Web sites / projects / products

\section{Conclusion}

\chapter{Appendices}
% Will be provided as separate tex documents and put afterwards, this is just
% for reminders
\section{Code selections}

\section{Transcripts}

\section{Requirement documents}

\section{Paper prototype(s)}

\end{document}
